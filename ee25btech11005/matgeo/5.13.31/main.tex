\documentclass{article}
\usepackage{amsmath}
\usepackage{amssymb}
\usepackage{gvv}
\usepackage{graphicx}
\usepackage{multicol}
\usepackage{enumitem}

\title{\textbf{5.13.31}}
\author{\textbf{Aditya Mishra — EE25BTECH11005}}
\date{October 11, 2025}

\begin{document}

\maketitle

\section*{Question}
Let \(\vec{A}\) and \(\vec{B}\) be \(3 \times 3\) matrices of real numbers, where \(\vec{A}\) is symmetric, \(\vec{B}\) is skew-symmetric and 
\[
(\vec{A} + \vec{B})(\vec{A} - \vec{B}) = (\vec{A} - \vec{B})(\vec{A} + \vec{B}).
\]
If 
\[
(\vec{A}\vec{B})^\top = (-1)^k \vec{A}\vec{B},
\]
where \((\vec{A}\vec{B})^\top\) is the transpose of the matrix \(\vec{A}\vec{B}\), find the value(s) of \(k\).

Given:
\begin{itemize}
    \item \(\vec{A}\) is symmetric: \(\vec{A}^\top = \vec{A}\)
    \item \(\vec{B}\) is skew-symmetric: \(\vec{B}^\top = -\vec{B}\)
\end{itemize}

\section*{Expanding the Given Equation}
Expand both sides of 
\[
(\vec{A} + \vec{B})(\vec{A} - \vec{B}) = (\vec{A} - \vec{B})(\vec{A} + \vec{B})
\]
Left:
\[
= \vec{A}^2 - \vec{A}\vec{B} + \vec{B}\vec{A} - \vec{B}^2
\]
Right:
\[
= \vec{A}^2 + \vec{A}\vec{B} - \vec{B}\vec{A} - \vec{B}^2
\]

\section*{Simplifying the Equation}
Set both expansions equal:
\[
\vec{A}^2 - \vec{A}\vec{B} + \vec{B}\vec{A} - \vec{B}^2 = \vec{A}^2 + \vec{A}\vec{B} - \vec{B}\vec{A} - \vec{B}^2
\]
Subtract \(\vec{A}^2\) and \(-\vec{B}^2\) from both sides:
\[
- \vec{A}\vec{B} + \vec{B}\vec{A} = \vec{A}\vec{B} - \vec{B}\vec{A}
\]
\[
2\vec{B}\vec{A} = 2\vec{A}\vec{B}
\]
\[
\implies \vec{A}\vec{B} = \vec{B}\vec{A}
\]
So \(\vec{A}\) and \(\vec{B}\) commute.

\section*{Transpose of the Product}
We have:
\[
(\vec{A}\vec{B})^\top = \vec{B}^\top\vec{A}^\top
\]
Using properties,
\[
= (-\vec{B})\vec{A} = -\vec{B}\vec{A}
\]
But since they commute:
\[
= -\vec{A}\vec{B}
\]
Thus,
\[
(\vec{A}\vec{B})^\top = -\vec{A}\vec{B}
\]

\section*{Solving for \(k\)}
Given,
\[
(\vec{A}\vec{B})^\top = (-1)^k\vec{A}\vec{B}
\]
So,
\[
- \vec{A}\vec{B} = (-1)^k \vec{A}\vec{B}
\]
If \(\vec{A}\vec{B} \neq 0\), comparing:
\[
-1 = (-1)^k
\]
So \(k\) must be odd.

\section*{\textbf{Final Answer}}
\[
\boxed{
k \text{ is any odd integer}.
}
\]

\end{document}

